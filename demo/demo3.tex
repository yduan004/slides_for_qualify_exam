%%%%%%%%%%%%%%%%%%%%%%%%%%%%%%%%%%
% Latex Beamer Slide Presentation %
%%%%%%%%%%%%%%%%%%%%%%%%%%%%%%%%%%%
% (1) Beamer installation
% Download the following 3 packages from 
%       http://sourceforge.net/project/showfiles.php?group_id=92412
% and save them in texmf tree of your home directory like this
% (a) latex-beamer goes in      ~/texmf/tex/latex/beamer 
% (b) pgf goes in               ~/texmf/tex/latex/pgf 
% (c) xcolor goes in            ~/texmf/tex/latex/xcolor

% (2) Beamer usage
% Manual for Beamer and Prosper: http://latex.perseguers.ch/contrib/presentations/guidelines.pdf
% Examples: http://latex-beamer.sourceforge.net/ 
% Example: http://www-verimag.imag.fr/~lmorel/html/beamer.html
% Manual: /home/tgirke/texmf/tex/latex/beamer/latex-beamer-3.06/doc/beameruserguide.pdf
% Print handouts: cp myslides.pdf zzz.pdf; pdftops -expand zzz.pdf; psnup -4 -b6mm -f zzz.ps > zzzhandouts.ps; ps2pdf zzzhandouts.ps
% generate PDF slide show with command:
% pdflatex demo3.tex; bibtex demo3.tex; pdflatex demo3.tex

\documentclass{beamer}
% Load a theme (graphics, colors,...) for the presentation
%\usepackage{beamerthemelined}
%\usepackage{beamerthemetree}
%\usetheme{default}
\usetheme{umbc2}
%\usetheme{EastLansing}
%\usepackage{beamerthemeclassic}
	
% For images:
\usepackage{graphicx}
% For color in text
\usepackage{color}

% For wrapping long URLs properly (may not be necessary)
\usepackage{url}

% Define comment command, which allows to comment out text with this syntax: \comment{my comment}
\newcommand{\comment}[1]{}

% Use UMBC theme collection. Download theme from: http://www.math.umbc.edu/~rouben/beamer/beamer-umbc.tar.gz
\useoutertheme{umbcfootline} 
% Define footnote line, see details: http://www.math.umbc.edu/~rouben/beamer/quickstart-Z-H-9.html#node_sec_9
\setfootline{\inserttitle \hfill \textit{\insertsection} \hfill \textit{\insertsubsection} \hfill Slide \insertframenumber/\inserttotalframenumber}

% BibTex Settings
\usepackage{natbib}
\renewcommand\refname{Bibliography} % Defines title of bibliography   

\hypersetup{pdfpagemode=FullScreen}
%%%%%%%%%%%%%%%%%%%%%%%%%%%%%%%%% SLIDE %%%%%%%%%%%%%%%%%%%%%%%%%%%%%%%%%
\title{Beamer Template}
\subtitle{A Quick Reference}
\author{Thomas Girke}
\date{\today}

% Manage internal and external links with hyperref (http://en.wikibooks.org/wiki/LaTeX/Hyperlinks)
\usepackage{hyperref}
\hypersetup{
	bookmarks=true,         % show bookmarks bar?  
	unicode=false,          % non-Latin characters in Acrobat'ss bookmarks 
	pdftoolbar=true,        % show Acrobat's toolbar?  
	pdfmenubar=true,        % show Acrobats menu?  
	pdffitwindow=false,     % window fit to page when opened
	pdfstartview={FitH},    % fits the width of the page to the window
	pdftitle={My title},    % title 
	pdfauthor={Author},     % author
	pdfsubject={Subject},   % subject of the document 
	pdfcreator={Creator},   % creator of the document 
	pdfproducer={Producer}, % producer of the document
	pdfkeywords={keywords}, % list of keywords 
	pdfnewwindow=true,      % links in new window 
	colorlinks=true,        % false: boxed links; true: colored links 
	linkcolor=blue,         % color of internal links 
	citecolor=blue,         % color of links to bibliography 
	filecolor=blue,         % color of file links 
	urlcolor=blue           % color of external links 
}

\begin{document}
\frame{\titlepage}
%%%%%%%%%%%%%%%%%%%%%%%%%%%%%%%%% SLIDE %%%%%%%%%%%%%%%%%%%%%%%%%%%%%%%%%
% Creates Separate Outline Slide at Beginning
%\section{Outline}
\frame{\tableofcontents}
%%%%%%%%%%%%%%%%%%%%%%%%%%%%%%%%% SLIDE %%%%%%%%%%%%%%%%%%%%%%%%%%%%%%%%%
% Define to generate outline slide automatically at start of every new section
\AtBeginSection[]
{
   \begin{frame}
       \frametitle{Outline}
       \tableofcontents[currentsection]
   \end{frame}
}
% Same effect at subsection level
%\AtBeginSubsection[]
%{
%   \begin{frame}
%       \frametitle{Outline}
%       \tableofcontents[currentsection,currentsubsection]
%   \end{frame}
%}
%%%%%%%%%%%%%%%%%%%%%%%%%%%%%%%%% SLIDE %%%%%%%%%%%%%%%%%%%%%%%%%%%%%%%%%
\section{Section 1}
%%%%%%%%%%%%%%%%%%%%%%%%%%%%%%%%% SLIDE %%%%%%%%%%%%%%%%%%%%%%%%%%%%%%%%%
\begin{frame}{Section 1}
\begin{itemize}
	\item Introduction Section 1
	\item ...
	\item ...
\end{itemize}
\end{frame}
%%%%%%%%%%%%%%%%%%%%%%%%%%%%%%%%% SLIDE %%%%%%%%%%%%%%%%%%%%%%%%%%%%%%%%%
\subsection{subsection 1}
%%%%%%%%%%%%%%%%%%%%%%%%%%%%%%%%% SLIDE %%%%%%%%%%%%%%%%%%%%%%%%%%%%%%%%%
\begin{frame}[allowframebreaks]{Section 1: subsection 1}
\begin{itemize}
	\item Long list with frame breaks 
	\item Long list with frame breaks 
	\item Long list with frame breaks 
	\item Long list with frame breaks 
	\item Long list with frame breaks 
	\item Long list with frame breaks 
	\item Long list with frame breaks 
	\item Long list with frame breaks 
	\item Long list with frame breaks 
	\item Long list with frame breaks 
	\item Long list with frame breaks 
	\item Long list with frame breaks 
	\item Long list with frame breaks 
	\item Long list with frame breaks 
	\item Long list with frame breaks 
	\item Long list with frame breaks 
	\item Long list with frame breaks 
	\item Long list with frame breaks 
	\item Long list with frame breaks 
	\item Long list with frame breaks 
\end{itemize}
\end{frame}
%%%%%%%%%%%%%%%%%%%%%%%%%%%%%%%%% SLIDE %%%%%%%%%%%%%%%%%%%%%%%%%%%%%%%%%
\subsection{subsection 2}
%%%%%%%%%%%%%%%%%%%%%%%%%%%%%%%%% SLIDE %%%%%%%%%%%%%%%%%%%%%%%%%%%%%%%%%
\begin{frame}{Section 1: subsection 2}
\begin{itemize}
	\item ...
	\item ...
	\item ...
\end{itemize}
\end{frame}
%%%%%%%%%%%%%%%%%%%%%%%%%%%%%%%%% SLIDE %%%%%%%%%%%%%%%%%%%%%%%%%%%%%%%%%
\section{Section 2}
%%%%%%%%%%%%%%%%%%%%%%%%%%%%%%%%% SLIDE %%%%%%%%%%%%%%%%%%%%%%%%%%%%%%%%%
\begin{frame}{Section 2}
\begin{itemize}
	\item Introduction Section 2
	\item ...
	\item ...
\end{itemize}
\end{frame}
%%%%%%%%%%%%%%%%%%%%%%%%%%%%%%%%% SLIDE %%%%%%%%%%%%%%%%%%%%%%%%%%%%%%%%%
\subsection{subsection 1}
%%%%%%%%%%%%%%%%%%%%%%%%%%%%%%%%% SLIDE %%%%%%%%%%%%%%%%%%%%%%%%%%%%%%%%%
\begin{frame}{Section 2: subsection 1}
\begin{itemize}
	\item ...
	\item ...
	\item ...
\end{itemize}
\end{frame}
%%%%%%%%%%%%%%%%%%%%%%%%%%%%%%%%% SLIDE %%%%%%%%%%%%%%%%%%%%%%%%%%%%%%%%%
\subsection{subsection 2}
%%%%%%%%%%%%%%%%%%%%%%%%%%%%%%%%% SLIDE %%%%%%%%%%%%%%%%%%%%%%%%%%%%%%%%%
\begin{frame}{Section 2: subsection 2}
\begin{itemize}
	\item ...
	\item ...
	\item ...
\end{itemize}
\end{frame}
%%%%%%%%%%%%%%%%%%%%%%%%%%%%%%%%% SLIDE %%%%%%%%%%%%%%%%%%%%%%%%%%%%%%%%%
\section{Section 3}
%%%%%%%%%%%%%%%%%%%%%%%%%%%%%%%%% SLIDE %%%%%%%%%%%%%%%%%%%%%%%%%%%%%%%%%
\begin{frame}{Section 3}
\begin{itemize}
	\item Introduction Section 3
	\item ...
	\item ...
\end{itemize}
\end{frame}
%%%%%%%%%%%%%%%%%%%%%%%%%%%%%%%%% SLIDE %%%%%%%%%%%%%%%%%%%%%%%%%%%%%%%%%
\subsection{subsection 1}
%%%%%%%%%%%%%%%%%%%%%%%%%%%%%%%%% SLIDE %%%%%%%%%%%%%%%%%%%%%%%%%%%%%%%%%
\begin{frame}{Section 3: subsection 1}
\begin{itemize}
	\item ...
	\item ...
	\item ...
\end{itemize}
\end{frame}
%%%%%%%%%%%%%%%%%%%%%%%%%%%%%%%%% SLIDE %%%%%%%%%%%%%%%%%%%%%%%%%%%%%%%%%
\subsection{subsection 2}
%%%%%%%%%%%%%%%%%%%%%%%%%%%%%%%%% SLIDE %%%%%%%%%%%%%%%%%%%%%%%%%%%%%%%%%
\begin{frame}{Section 3: Citations with BibTex Support}
\begin{itemize}
	\item Citation in parentheses \citep{Knuth92, ConcreteMath, Simpson}
	\item Citation of \cite{Knuth92, ConcreteMath, Simpson}
	\item Extended citation \citep[][J Chem Inf Model, 46, 1912-1918]{ConcreteMath}
	\item Footnote citation with more detail \citep{Knuth92}\footnote{\citep[][J Chem Inf Model]{Simpson}}
	\item ...
\end{itemize}
\end{frame}
%%%%%%%%%%%%%%%%%%%%%%%%%%%% REFERENCE LIST %%%%%%%%%%%%%%%%%%%%%%%%%%%%%
\def\newblock{\hskip .11em plus .33em minus .07em} % Important line to support typical BibTex behavior in Beamer slide presentation
\begin{frame}[allowframebreaks]{Bibliography}
\scriptsize 
\bibliographystyle{elsart-harv.bst} % Uses style file "elsart-harv.bst" AND requires in preamble \usepackage{natbib}; many more styles can be found at http://jo.irisson.free.fr/bstdatabase/
\bibliography{demo.bib} % Expects bibliography file "MyBibTex.bib"
\nocite{Knuth92, ConcreteMath, Simpson} % includes selected references without citing them
% \nocite{*} % includes all references from a bibtex database
\end{frame}
%%%%%%%%%%%%%%%%%%%%%%%%%%%% REFERENCE LIST %%%%%%%%%%%%%%%%%%%%%%%%%%%%%

\end{document}

